\section{Introduction} \label{sec:Introduction}

\input{figs/structure_fig}

\subsection{Importance of OOD Detection} \label{subsec:Importance of OOD Detection}

Out-of-Distribution (OOD) detection is crucial for developing reliable AI systems, safeguarding against erroneous predictions from inputs that deviate from training data distributions \cite{HowGoodAre3,Delvingint2}. This capability is essential for AI models to perform effectively in real-world scenarios with novel data. The vulnerability of machine learning models to adversarial attacks highlights the need for robust detection mechanisms that extend beyond adversarial examples \cite{arous2023noiseyouadversarialtraining}.

In natural language processing (NLP), OOD detection enhances reliability by identifying inputs that diverge from the training distribution \cite{lang2023survey}. The rise of generative AI models, particularly conversational agents, amplifies the necessity for effective OOD detection to mitigate risks of misinformation and manipulation, as AI-generated content often closely resembles human-generated text \cite{islam2023distinguishinghumangeneratedtext}.

Furthermore, hidden incentives for auto-induced distributional shifts (HI-ADS) present ethical and performance challenges, underscoring OOD detection's role in addressing unintended consequences of AI systems influencing input distributions \cite{krueger2020hiddenincentivesautoinduceddistributional}. The detection of deepfakes also exemplifies the necessity of OOD detection in maintaining trust in digital content \cite{cao2021understandingsecuritydeepfakedetection}.

Integrating uncertainty quantification methods into OOD detection frameworks is vital for developing AI systems that provide confidence measures in predictions, enhancing decision-making \cite{walz2023easyuncertaintyquantificationeasyuq}. Additionally, improving stochastic nonconvex optimization algorithms in the presence of outliers is crucial for AI model robustness, emphasizing OOD detection's importance in refining these methodologies \cite{li2024robustsecondordernonconvexoptimization}.

In healthcare, non-invasive and interpretable machine learning approaches are critical, particularly in cancer diagnosis, where OOD detection ensures the accuracy and trustworthiness of models \cite{an2023intelligentdiagnosticschemelung}. In wireless communications, OOD detection protects information privacy by preventing unauthorized access and eavesdropping \cite{kim2020make5gcommunicationsinvisible}.

Ultimately, OOD detection is vital for enhancing AI systems, enabling them to identify unknown categories while maintaining performance on known classes, leading to more secure, reliable, and adaptable models \cite{lang2023survey,Delvingint2}. By addressing current methods' limitations, OOD detection significantly contributes to AI systems' overall trustworthiness across diverse applications.

\subsection{Interrelated Concepts} \label{subsec:Interrelated Concepts}

The landscape of OOD detection is intricately linked with key concepts in machine learning, enhancing AI robustness and reliability. Anomaly detection focuses on identifying rare patterns within the same distribution \cite{lang2023survey}, while OOD detection identifies inputs outside the learned distribution, improving model handling of unforeseen data \cite{Out-of-Dis1}.

Adversarial examples, crafted to mislead models, challenge their robustness and security. Integrating adversarial machine learning (AML) into corporate workflows is crucial for addressing these vulnerabilities \cite{kim2020make5gcommunicationsinvisible}. Adversarial training techniques and PAC confidence sets are pivotal for ensuring models provide actionable recourses with high probability, fortifying them against adversarial manipulations \cite{cao2021understandingsecuritydeepfakedetection}.

Uncertainty estimation is vital in dynamic environments, necessitating trustworthy systems that can externalize uncertainty, particularly in language models \cite{islam2023distinguishinghumangeneratedtext}. This fosters effective human-machine collaboration and adaptability to new scenarios \cite{Delvingint2}.

In NLP, synthetic data generated by advanced models enhances classification performance, addressing challenges posed by semantic shifts and out-of-domain inputs \cite{lang2023survey}. 

The interplay between these concepts necessitates a holistic approach to OOD detection, treating generalization and detection as intertwined challenges. This integrated perspective is essential for developing robust AI systems capable of maintaining trustworthiness against adversarial threats and backdoor attacks \cite{HowGoodAre3}.

\subsection{Structure of the Survey} \label{subsec:Structure of the Survey}

This survey is structured to provide a comprehensive understanding of OOD detection in machine learning, particularly its applications in NLP. It begins with the importance of OOD detection, followed by interrelated concepts like anomaly detection, adversarial examples, and uncertainty estimation. Subsequent sections delve into core concepts, defining key terms and their significance in the broader context of machine learning.

The survey examines various OOD detection techniques, categorizing them based on OOD data availability, including density estimation, distance-based approaches, and deep learning techniques, highlighting their strengths and limitations. Special attention is given to innovative methods emerging from recent research.

In NLP, the survey addresses specific challenges, such as semantic shifts and out-of-domain inputs. It assesses the robustness of various embedding models and establishes a benchmark for evaluating NLP model performance in OOD scenarios, ensuring chosen datasets exhibit low semantic similarity for effective evaluation \cite{yuan2023revisiting}. It also analyzes anomaly detection and adversarial examples, exploring their overlap with OOD detection and impact on machine learning robustness.

The survey further investigates uncertainty estimation methods, emphasizing their relevance to OOD detection and secure AI systems. Finally, it identifies current challenges and future directions in OOD detection, highlighting emerging trends and technological advancements shaping the field. This structured approach facilitates a thorough exploration of OOD detection, offering valuable insights for researchers and practitioners alike \cite{lang2023survey}.


The following sections are organized as shown in \autoref{fig:chapter_structure}.








\section{Background and Core Concepts} \label{sec:Background and Core Concepts}

\subsection{Out-of-Distribution Detection} \label{subsec:Out-of-Distribution Detection}

Out-of-Distribution (OOD) detection is crucial in machine learning for identifying data points that fall outside the learned distribution from training datasets, essential for building robust AI systems capable of managing unpredictable inputs effectively \cite{Out-of-Dis1}. The complexity of this task lies in distinguishing between in-distribution and OOD inputs, particularly in high-dimensional spaces where subtle differences can be challenging to detect \cite{zhao2021suodacceleratinglargescaleunsupervised}.

In applications such as spoken language understanding (SLU), models encounter OOD challenges like out-of-vocabulary (OOV) terms and compositional generalization (CG) \cite{porjazovski2024outofdistributiongeneralisationspokenlanguage}. Despite advances in OOD detection, a comprehensive framework integrating these aspects is still lacking, hindering both theoretical insights and practical applications \cite{wang2024bridgingooddetectiongeneralization}.

The significance of OOD detection spans multiple fields. In cybersecurity, it helps identify backdoor attacks exploiting model vulnerabilities \cite{pichler2024infeasibilitymlbackdoordetection}. In medical imaging, it aids in recognizing outlier images to prevent misdiagnoses \cite{islam2024outlierdetectionlargeradiological}. Furthermore, the tendency of neural networks to exhibit overconfidence in predictions complicates reliable classifications, necessitating methods that detect OOD inputs while calibrating confidence levels \cite{HowGoodAre3,li2021automateddataengineeringpipeline}.

In automated data engineering, sophisticated OOD detection mechanisms are required to handle anomalies in complex IoT sensor data \cite{li2019repairremovingrepresentationbias} and address hidden incentives for auto-induced distributional shifts (HI-ADS) in machine learning models \cite{krueger2020hiddenincentivesautoinduceddistributional}. The detection of visually similar OOD samples is crucial for adversarial robustness, especially in fields like deepfake detection and healthcare diagnostics \cite{Delvingint2,cao2021understandingsecuritydeepfakedetection,an2023intelligentdiagnosticschemelung}.

The sensitivity of Wasserstein distances to outliers necessitates robust OOD detection techniques \cite{nietert2023outlierrobustoptimaltransportduality}, while outlier-aware network embedding (ONE) algorithms demonstrate the breadth of OOD detection applications \cite{bandyopadhyay2018outlierawarenetworkembedding}. Overall, OOD detection is essential for enhancing the security, robustness, and trustworthiness of AI systems, paving the way for generalizable and reliable machine learning applications \cite{baragatti2024approximatebayesiancomputationdeep}.

\subsection{Natural Language Processing} \label{subsec:NLP}

NLP plays a critical role in OOD detection, particularly in enhancing the robustness of language models. Natural Language Understanding (NLU) is vital for accurately interpreting user inputs, which often contain OOD elements \cite{Out-of-dom4}. The challenge of limited genuine datasets complicates training NLP classification models, adversely affecting predictive performance \cite{whitfield2021usinggpt2createsynthetic}.

Synthetic data generation has emerged as a promising solution to augment training datasets, thereby enhancing models' ability to handle OOD inputs. For instance, a balanced dataset of 10,000 texts—split between human-written and ChatGPT-generated content—provides a robust foundation for classification tasks \cite{islam2023distinguishinghumangeneratedtext}. This diversity facilitates more effective OOD detection.

The absence of standardized frameworks for evaluating embedding models hampers the assessment of their robustness against OOD inputs \cite{darrin2024embeddingmodelpromisinganother}. Innovative techniques like the Maximum Concept Matching (MCM) method, which aligns visual features with textual prototypes, exemplify advancements in enhancing OOD detection capabilities \cite{Delvingint2}.

The scarcity of representative OOD data is critical for accurately estimating OOD distributions, ensuring NLP models can generalize effectively beyond training data \cite{lang2023survey}. Addressing this gap is vital for developing NLP systems that are robust, reliable, and trustworthy when facing novel inputs.

\subsection{Anomaly Detection} \label{subsec:Anomaly Detection}

Anomaly detection is pivotal in machine learning, focusing on identifying observations that deviate from expected patterns across various domains, including finance, cybersecurity, and healthcare \cite{frank2024circuitdesignbiologymachine}. This well-established field employs techniques to detect both low-level and high-level anomalies in complex datasets \cite{boersma2024flexiblecategorizationusingformal}. Anomalies often indicate significant, albeit rare, events such as fraud or system failures.

Anomaly detection and OOD detection share a foundational overlap, as both aim to identify instances deviating from learned distributions. While anomaly detection traditionally focuses on rare events within a distribution, OOD detection identifies inputs outside the training distribution, enhancing models' capacity to handle unforeseen data \cite{li2019anomalydetectiongenerativeadversarial}.

Detecting anomalies in multivariate time series data, especially in Cyber-Physical Systems (CPSs) during cyber-attacks, underscores the complexity of the task \cite{li2019anomalydetectiongenerativeadversarial}. In IoT sensor data, automated data engineering pipelines integrating machine learning techniques have been proposed to efficiently identify anomalies, ensuring data integrity \cite{li2021automateddataengineeringpipeline}.

In financial auditing, inadequate categorization of business processes complicates anomaly detection, necessitating sophisticated approaches \cite{boersma2024flexiblecategorizationusingformal}. The limitations of Property Specification Language (PSL) in handling continuous data further challenge real-time anomaly detection scenarios \cite{smith2022psldeadlonglive}.

In the energy sector, unsupervised anomaly detection optimizes resource usage during peak demand periods \cite{pallage2024slicedwassersteinbasedanomalydetectionopen}. Additionally, benchmarks for identifying outlier images in radiology enhance the reliability of machine learning models in clinical applications \cite{islam2024outlierdetectionlargeradiological}.

The presence of outlier nodes in attributed networks disrupts node embedding consistency, emphasizing the importance of anomaly detection \cite{bandyopadhyay2018outlierawarenetworkembedding}. The development of minimal circuits for cellular-scale operations highlights the need for tailored anomaly detection solutions \cite{frank2024circuitdesignbiologymachine}.

Overall, the intersection of anomaly detection and OOD detection underscores the necessity for robust frameworks capable of addressing both in-distribution anomalies and out-of-distribution inputs, essential for developing secure and reliable AI systems that generalize effectively in dynamic environments.

\subsection{Adversarial Examples} \label{subsec:Adversarial Examples}

Adversarial examples significantly challenge machine learning, undermining model robustness and security across diverse tasks, such as classification and object detection. Recent research explores the relationship between model interpretability and adversarial robustness, emphasizing the need to understand and mitigate these vulnerabilities \cite{liu2020adversarialattacksdefensesinterpretation}. These examples involve inputs altered by small perturbations to deceive models into incorrect predictions.

Adversarial examples often reside in low-probability regions of the training distribution, complicating detection and counteraction \cite{song2018pixeldefendleveraginggenerativemodels}. This characteristic necessitates robust detection mechanisms to identify and mitigate adversarial inputs. Innovative algorithms, such as FilTrun, utilize filtration and truncation strategies to enhance model robustness against specific attacks \cite{delgosha2021robustclassificationell0attack}.

The phenomenon of Neural Collapse, which describes the alignment of feature representations in deep neural networks, has been examined concerning adversarial robustness. Understanding adversarial attacks' impact on these geometric properties is crucial for developing resilient models \cite{su2024robustnessneuralcollapseneural}.

Current research often lacks standardized definitions and evaluations of adversarial examples, leading to inconsistencies and complicating effective defenses \cite{Out-of-Dis0}. This gap highlights the need for clear benchmarks and methodologies for assessing adversarial robustness across models and attack scenarios.

Various defense mechanisms have been proposed in response to adversarial threats, including adversarial training, which augments training data with adversarial examples to improve model resilience \cite{arous2023noiseyouadversarialtraining}. Integrated approaches combining autoencoder techniques with block switching show promise in mitigating adversarial perturbations, contributing to overall model robustness \cite{yadav2022integratedautoencoderblockswitching}.

Addressing adversarial examples is essential for enhancing AI model reliability and security. By integrating robust detection and defense strategies with interpretability analyses, researchers can improve machine learning systems' resilience against adversarial attacks, fostering the development of trustworthy AI applications \cite{anisetti2023rethinkingcertificationtrustworthymachine,liu2020adversarialattacksdefensesinterpretation}.

\subsection{Uncertainty Estimation} \label{subsec:Uncertainty Estimation}

Uncertainty estimation is crucial for developing robust machine learning models, providing confidence measures in predictions and facilitating informed decision-making. The ε-outlier-robust Wasserstein distance enhances uncertainty quantification by excluding a fraction of outlier mass from distributions, improving reliability in the presence of outliers \cite{nietert2023outlierrobustoptimaltransportduality}.

Integrating multimodal data-driven prior distributions, such as the m2d2 method, enriches uncertainty estimation by capturing complexities inherent in multimodal data, leading to improved predictive reliability \cite{lopez2023informativepriorsimprovereliability}.

Incorporating user behaviors into uncertainty estimation frameworks is another promising avenue. The SyRoUP method accounts for user interactions, refining uncertainty estimation by acknowledging the dynamic nature of user-model interactions \cite{sicilia2024accountingsycophancylanguagemodel}.

Furthermore, quantifying both confidence and atypicality enhances the reliability of machine learning models. By separately assessing prediction confidence and input atypicality, models can provide comprehensive uncertainty estimates, improving robustness across diverse scenarios \cite{yuksekgonul2023confidencereliablemodelsconsider}.

Adversarial training, which augments datasets with adversarial examples, bolsters model robustness and contributes to reliable uncertainty estimation by exposing models to challenging inputs \cite{debicha2021adversarialtrainingdeeplearningbased}.

Logit Normalization addresses overconfidence in neural networks by decoupling the influence of the logit's norm from training objectives, leading to more calibrated uncertainty estimates \cite{wei2022mitigatingneuralnetworkoverconfidence}.

Integrating these methods into machine learning workflows is essential for developing models that effectively quantify uncertainty, facilitating informed decision-making and enhancing AI systems' overall trustworthiness.











\section{Out-of-Distribution Detection Techniques} \label{sec:Out-of-Distribution Detection Techniques}


Enhancing the robustness of machine learning models necessitates a thorough exploration of Out-of-Distribution (OOD) detection techniques. This section organizes OOD detection methods into a structured framework based on the availability of OOD data, facilitating a systematic evaluation of diverse methodologies. As illustrated in \autoref{fig:tree_figure_Out-o}, the hierarchical classification of OOD detection techniques is organized into five primary categories: Taxonomy of OOD Detection Methods, Density Estimation Techniques, Distance-Based and Graph-Theoretic Approaches, Deep Learning and Neural Network-Based Techniques, and Innovative Methods and Frameworks. Each category is further subdivided into specific methods and their respective contributions to OOD detection, which underscores the diverse methodologies employed to enhance the robustness and reliability of machine learning models. The following subsection delves into these categories, emphasizing their implications for future research and practical applications.

\input{figs/tree_figure_Out-o}



\subsection{Taxonomy of OOD Detection Methods} \label{subsec:Taxonomy of OOD Detection Methods}

A comprehensive understanding of OOD detection methods is crucial for improving machine learning model reliability. A detailed taxonomy classifies these methods based on the availability of OOD data into three categories: OOD data available, OOD data unavailable with in-distribution (ID) labels, and OOD data unavailable with no labels \cite{lang2023survey}. This structured framework aids in assessing the strengths and limitations of each approach.

As illustrated in \autoref{fig:tiny_tree_figure_0}, the taxonomy of out-of-distribution (OOD) detection methods categorizes them based on the availability of OOD data and ID labels. It highlights key techniques employed when OOD data is available, such as feature representation and generalization capabilities, alongside methods like density estimation and distance-based techniques utilized when OOD data is absent but ID labels are present. Furthermore, in scenarios where both OOD data and labels are unavailable, unsupervised approaches rely on intrinsic data properties to identify anomalies.

This taxonomy not only facilitates systematic classification but also serves as a foundation for innovative strategies tailored to address unique OOD detection challenges \cite{lang2023survey}. By clarifying data availability, it fosters a nuanced understanding of the trade-offs and applicability of different techniques, ultimately advancing the development of more robust AI systems.

\input{figs/tiny_tree_figure_0}
\subsection{Density Estimation Techniques} \label{subsec:Density Estimation Techniques}

Density estimation techniques are pivotal in OOD detection, modeling the probability distribution of in-distribution data to identify deviations indicative of OOD instances. These methods operate on the premise that OOD data points often reside in low-density areas of the learned distribution, which is essential for developing robust systems capable of recognizing unknown classes \cite{lang2023survey,Delvingint2}.

The Sliced-Wasserstein Filter (SWAD) identifies outliers in energy consumption data using the sliced-Wasserstein distance \cite{pallage2024slicedwassersteinbasedanomalydetectionopen}. By projecting high-dimensional data into one-dimensional slices, it facilitates efficient anomaly detection. The Wεp technique improves density estimation by removing an ε fraction of outlier mass from distributions before calculating the Wasserstein distance, enhancing the accuracy of OOD detection \cite{nietert2023outlierrobustoptimaltransportduality}.

The Uniform Manifold Approximation and Projection (UMAP) algorithm exemplifies the utility of density estimation by clustering similar images and identifying outliers in large radiological datasets \cite{islam2024outlierdetectionlargeradiological}. Overall, density estimation techniques provide a probabilistic foundation for OOD detection, particularly through Bayesian inference, enhancing the ability to differentiate between in-distribution and OOD data. This is crucial in NLP, where unique challenges must be addressed to improve detection accuracy \cite{lang2023survey,Out-of-Dis0}. By refining these methods, researchers can enhance model generalization, contributing to more reliable AI systems.

\subsection{Distance-Based and Graph-Theoretic Approaches} \label{subsec:Distance-Based and Graph-Theoretic Approaches}

Distance-based and graph-theoretic approaches offer a unique perspective in OOD detection by leveraging spatial relationships among data points to identify anomalies. Our framework provides a closed-form solution to quantify distances between in-distribution (ID) and semantic OOD data, enhancing understanding of OOD detection performance (Section 4). Concept matching is also employed to derive the OOD detector G(·) \cite{wang2024bridgingooddetectiongeneralization,Delvingint2}.

A notable contribution is the graph-theoretic framework by Wang et al., which addresses OOD detection and generalization challenges \cite{wang2024bridgingooddetectiongeneralization}. This approach constructs a graph where nodes represent data points and edges denote similarity, effectively identifying OOD instances through weak or anomalous connections.

Distance metrics, such as Euclidean or Mahalanobis distances, quantify similarity between data points, forming the basis for various OOD detection techniques. OOD data points typically exhibit larger distances from ID data centroids, facilitating effective concept matching \cite{lang2023survey,Delvingint2}. Graph-theoretic approaches extend this by analyzing connectivity and path lengths to discern OOD samples.

Integrating graph-based methods with traditional distance metrics enhances OOD detection performance in high-dimensional spaces. This approach provides a nuanced understanding of data complexities while offering a closed-form solution for distance quantification, ultimately improving anomaly detection \cite{wang2024bridgingooddetectiongeneralization}.

\subsection{Deep Learning and Neural Network-Based Techniques} \label{subsec:Deep Learning and Neural Network-Based Techniques}

Deep learning techniques are central to advancing OOD detection, utilizing their ability to model complex data distributions and enhance robustness against adversarial threats. Adversarial training integrates adversarial examples into the training process, bolstering model resilience, particularly in applications like anomaly-based intrusion detection \cite{arous2023noiseyouadversarialtraining}.

The Latent Poisoning method targets internal representations of classifiers, highlighting the importance of manipulating latent spaces for improved OOD detection \cite{creswell2017latentpoisonadversarialattacks}. Additionally, embedding Gaussian noise within neural network layers creates a stochastic model better equipped to handle adversarial perturbations \cite{arous2023noiseyouadversarialtraining}.

The Tensor-Network Machine Learning (TN-ML) method classifies lung cancer patients based on Raman spectra data, showcasing the application of deep learning in healthcare OOD detection \cite{an2023intelligentdiagnosticschemelung}. Moreover, cosine distance integration in large language models (LLMs) effectively distinguishes between in-distribution and OOD inputs, enhancing generalization capabilities \cite{HowGoodAre3}.

The Adversarial Perturbation for Cooperative Jamming (CJ-AP) method demonstrates deep learning's versatility in OOD detection within wireless communications \cite{kim2020make5gcommunicationsinvisible}. Overall, deep learning techniques provide a robust foundation for OOD detection, offering innovative solutions to enhance model resilience and reliability against adversarial threats and unforeseen data \cite{debicha2021adversarialtrainingdeeplearningbased}.

\subsection{Innovative Methods and Frameworks} \label{subsec:Innovative Methods and Frameworks}

Recent advancements in OOD detection have introduced innovative methods and frameworks that enhance the robustness and reliability of machine learning models. The Outlier-Robust Second-Order Optimization (ORSO) method finds approximate second-order stationary points in stochastic optimization problems with corrupted data, improving model resilience against outliers \cite{li2024robustsecondordernonconvexoptimization}.

The Outlier-Aware Network Embedding (ONE) method incorporates outlier effects into attributed network embeddings, facilitating more accurate OOD detection in complex network structures \cite{bandyopadhyay2018outlierawarenetworkembedding}. The Approximate Bayesian Computation with Deep Learning (ABCD-Conformal) framework provides confidence intervals with guaranteed marginal coverage, enhancing uncertainty estimation in OOD detection \cite{baragatti2024approximatebayesiancomputationdeep}.

The SUOD system focuses on data reduction and model approximation to improve performance in outlier detection, particularly in large-scale unsupervised settings \cite{zhao2021suodacceleratinglargescaleunsupervised}. The REPAIR method minimizes representation bias by optimizing example-level weights in datasets, enhancing fairness and accuracy in OOD detection \cite{li2019repairremovingrepresentationbias}.

Additionally, Wεp exemplifies advancements in reliable distance measurements for OOD detection by mitigating outlier influence \cite{nietert2023outlierrobustoptimaltransportduality}. These innovative methods and frameworks significantly enhance the security, reliability, and adaptability of AI systems, enabling effective detection of out-of-distribution samples. This capability is crucial for ensuring the safety of applications like text classification, question answering, and machine translation, ultimately advancing the performance and robustness of NLP systems \cite{lang2023survey,Delvingint2}.











\section{Applications in NLP} \label{sec:Applications in NLP}

The role of Out-of-Distribution (OOD) detection in NLP is pivotal for enhancing model robustness and adaptability. This section delves into the specific challenges faced in OOD detection for NLU, highlighting the complexities that researchers must navigate.

\subsection{Challenges in OOD Detection for NLU} \label{subsec:Challenges in OOD Detection for NLU}

OOD detection in NLU is hindered by the dependency on expensive, manually labeled OOD samples, which restricts the availability of comprehensive datasets for training robust models \cite{Out-of-dom4}. The observed performance degradation in models when encountering OOD splits underscores the difficulty in generalizing across diverse linguistic inputs \cite{porjazovski2024outofdistributiongeneralisationspokenlanguage}. Evaluating NLP models under distribution shifts, such as semantic and background variations, adds layers of complexity to OOD detection \cite{yuan2023revisiting}. Addressing these shifts with tailored benchmarks is crucial for developing models adaptable to dynamic linguistic environments.

As illustrated in \autoref{fig:tiny_tree_figure_1}, the main challenges in out-of-distribution (OOD) detection for natural language understanding (NLU) can be categorized into three primary areas: dependency on labeled data, performance degradation, and the need for efficient detection methods. Each category highlights specific issues, such as the reliance on expensive OOD samples, difficulties in generalizing across distribution shifts, and the demand for efficient OOD detection methods in embedded systems.

Moreover, the demand for efficient OOD detection methods in embedded systems highlights the need to balance memory usage, execution time, and detection accuracy \cite{bansal2024compressingvaebasedoutofdistributiondetectors}. Techniques like Logit Normalization (LogitNorm) improve the separability between in-distribution and OOD inputs, enhancing detection rates and reducing neural network overconfidence \cite{wei2022mitigatingneuralnetworkoverconfidence}. Comprehensive benchmarks for evaluating AI-generated content, such as phishing emails, are necessary to integrate various text analysis methods and machine learning tools effectively, thereby identifying misinformation \cite{eze2024analysispreventionaibasedphishing}. These benchmarks are essential for assessing model performance in detecting AI-generated anomalies and improving robustness against OOD inputs.

Representation bias in datasets further complicates the fairness and generalization of OOD detection models. Techniques like REPAIR, which aim to minimize representation bias, are crucial for ensuring equitable evaluations of algorithm performance \cite{li2019repairremovingrepresentationbias}. Addressing these challenges requires a multifaceted approach involving efficient data collection, robust evaluation frameworks, and innovative detection techniques. By tackling the unique challenges in OOD detection for NLU—such as discrete input spaces and contextual nuances—researchers can significantly enhance the reliability of language models, which is vital for developing effective NLP systems across various applications like text classification, question answering, and machine translation \cite{lang2023survey,HowGoodAre3}.
\input{figs/tiny_tree_figure_1}
\subsection{Synthetic Data Generation for Enhanced Classification} \label{subsec:Synthetic Data Generation for Enhanced Classification}

Synthetic data generation has emerged as a crucial strategy for improving OOD detection in NLP, especially when obtaining large-scale genuine datasets is impractical. Utilizing models like GPT-2 for synthetic data reduces the need for extensive data collection, lowering costs while enhancing model performance with limited genuine data \cite{whitfield2021usinggpt2createsynthetic}. This approach is particularly beneficial in scenarios where labeled OOD samples are scarce, allowing for the creation of diverse datasets that bolster NLP model robustness.

Recent studies emphasize the need for standardized benchmarks for evaluating OOD robustness, advocating for a consistent framework to assess the effectiveness of synthetic data in improving model generalization \cite{yuan2023revisiting}. These benchmarks facilitate comprehensive evaluations of NLP models across various distribution shifts, ensuring that synthetic data integration significantly enhances OOD detection capabilities.

By leveraging synthetic data generation, researchers can overcome the limitations of genuine datasets, equipping NLP models with the variability needed to handle unforeseen inputs effectively. This strategy not only enhances robustness and adaptability but also fosters the development of reliable AI systems capable of operating in dynamic linguistic environments.

\subsection{Evaluation of Embedding Models Across NLP Tasks} \label{subsec:Evaluation of Embedding Models Across NLP Tasks}

Evaluating embedding models across NLP tasks is essential for understanding their robustness against OOD inputs. A recent comparison of 34 different embedding models across multiple datasets and downstream tasks reveals their diverse capabilities in managing linguistic variability \cite{darrin2024embeddingmodelpromisinganother}. This extensive evaluation provides insights into the strengths and weaknesses of various embeddings, aiding in the selection of models that maintain performance across different contexts.

The introduction of the BOSS benchmark marks a significant advancement in embedding model evaluation, offering a comprehensive suite that covers multiple tasks and datasets with distinct distribution shifts \cite{yuan2023revisiting}. This benchmark improves upon previous evaluations limited to single tasks, enabling a holistic assessment of model robustness. By addressing various distribution shifts, BOSS allows researchers to systematically evaluate how well embedding models generalize to OOD scenarios, contributing to the development of resilient NLP systems.

These evaluations help identify embedding models that exhibit strong generalization capabilities and effectively separate in-distribution (ID) data from semantic OOD data. This focus on robustness is crucial for advancing NLP technologies and enhancing the reliability of language models in unpredictable environments \cite{wang2024bridgingooddetectiongeneralization,darrin2024embeddingmodelpromisinganother}.

\subsection{Benchmarking OOD Robustness in NLP Models} \label{subsec:Benchmarking OOD Robustness in NLP Models}

Benchmarking the robustness of NLP models in OOD scenarios is vital for assessing their generalization capabilities across diverse linguistic inputs. Recent efforts have compiled challenging datasets from resources like Paperswithcode, Kaggle, and the ACL Anthology \cite{yuan2023revisiting}. This approach ensures that benchmarks reflect a wide range of scenarios, enabling comprehensive evaluations of model performance under various distribution shifts.

Establishing standardized evaluation benchmarks is crucial for systematically assessing OOD robustness, addressing the lack of uniform datasets specifically designed for this purpose, which has led to reliance on heuristic methods in prior research \cite{yuan2023revisiting}. These benchmarks facilitate comparisons among models across multiple tasks and datasets, revealing strengths and weaknesses in handling OOD inputs. A structured evaluation framework allows researchers to identify models with superior generalization capabilities, ensuring their applicability in real-world scenarios.

Overall, developing robust benchmarking methodologies is essential for advancing NLP, fostering the creation of models that are not only accurate but also resilient to OOD challenges. Through rigorous evaluation, researchers can drive the innovation of reliable and adaptable language models, enhancing the effectiveness of AI systems in dynamic environments.










\section{Anomaly Detection and Adversarial Examples} \label{sec:Anomaly Detection and Adversarial Examples}

The interplay between anomaly detection and adversarial examples is pivotal in enhancing the robustness of machine learning models. As models encounter both benign variations and malicious perturbations in practical scenarios, understanding the role of anomaly detection in identifying adversarial inputs is crucial for system dependability. The following subsection delves into the connection between anomaly detection and Out-of-Distribution (OOD) detection, emphasizing shared methodologies and their implications for safeguarding models against adversarial challenges.


\subsection{Overlap between Anomaly Detection and OOD Detection} \label{subsec:Overlap between Anomaly Detection and OOD Detection}

Anomaly detection and OOD detection both aim to identify data points that deviate from expected patterns, though they focus on different aspects. Anomaly detection targets atypical patterns within the same distribution, which is essential for applications like fraud detection and network security, whereas OOD detection identifies inputs outside the learned distribution, augmenting models' capability to manage unforeseen data \cite{lang2023survey}.

An example of this synergy is the ONE, which incorporates outlier detection into the embedding process to enhance the accuracy of normal node representations \cite{bandyopadhyay2018outlierawarenetworkembedding}. This integration highlights the common techniques between anomaly and OOD detection, particularly in network analysis. \autoref{fig:tiny_tree_figure_2} illustrates the overlap between anomaly detection and out-of-distribution (OOD) detection, highlighting key focus areas, techniques, and applications. It emphasizes the shared goal of identifying data deviations, with specific examples like ONE and adversarial attacks, and their relevance in network analysis and NLP tasks.

The susceptibility of models to adversarial attacks, as demonstrated by black-box adversarial methods, underscores the necessity for robust detection mechanisms. A single adversarially perturbed sample can distort decision boundaries, leading to the misclassification of multiple unperturbed samples \cite{chhabra2019suspicionfreeadversarialattacksclustering}. This vulnerability is relevant to both anomaly and OOD detection, stressing the importance of models being able to identify perturbations indicative of anomalous or out-of-distribution inputs.

Lang et al.'s survey further investigates these similarities and differences, especially in NLP tasks, where issues like semantic shifts and out-of-domain inputs frequently occur \cite{lang2023survey}. By exploring these overlaps, researchers can develop comprehensive frameworks that leverage the strengths of both anomaly and OOD detection, ultimately enhancing the robustness of AI systems.

\input{figs/tiny_tree_figure_2}
\subsection{Impact on Machine Learning Robustness} \label{subsec:Impact on Machine Learning Robustness}

Adversarial examples pose a significant threat to the robustness of machine learning models, often causing incorrect predictions with minimal perturbations. These inputs, crafted to mimic legitimate data, can result in misclassification, thereby compromising model reliability and security \cite{cao2021understandingsecuritydeepfakedetection}. The vulnerability of deep neural networks to such attacks necessitates the development of robust detection and defense mechanisms \cite{chhabra2019suspicionfreeadversarialattacksclustering}.

Adversarial training offers a strategic advantage by enabling robust model training without the computational burden of generating adversarial examples, all while preserving high accuracy \cite{arous2023noiseyouadversarialtraining}. However, providing an accurate statistical analysis of the sparsity-recovery phenomenon in adversarial training under $\ell_\infty$-perturbations remains a challenge \cite{xie2024asymptoticbehavioradversarialtraining}.

The impact of adversarial examples spans numerous domains, including healthcare and communications. For instance, the TN-ML method enhances interpretability and accuracy in lung cancer screening, addressing adversarial effects on model robustness \cite{an2023intelligentdiagnosticschemelung}. In wireless communications, adversarial perturbations compromise the robustness of models used by eavesdroppers to detect transmissions \cite{kim2020make5gcommunicationsinvisible}.

Moreover, adversarial attacks can distort decision boundaries in clustering algorithms \cite{chhabra2019suspicionfreeadversarialattacksclustering}. This vulnerability necessitates effective defenses to safeguard machine learning systems against adversarial threats.

Addressing adversarial examples is essential for bolstering AI model reliability and security. This involves developing robust methodologies for generating adversarial samples across various domains—such as image, text, tabular, and graph data—while integrating interpretability into analyses of adversarial robustness \cite{liu2020adversarialattacksdefensesinterpretation}. By establishing robust detection and defense strategies, researchers can enhance machine learning systems' resilience to adversarial attacks, leading to more trustworthy AI applications.

\subsection{Enhancing Robustness Against Adversarial Examples} \label{subsec:Enhancing Robustness Against Adversarial Examples}

Enhancing the robustness of machine learning models against adversarial examples is vital for ensuring their reliability and security. VERIGB represents a promising approach by formalizing the verification process for gradient-boosted models, offering a structured framework to verify model prediction integrity and mitigate adversarial impacts \cite{einziger2019verifyingrobustnessgradientboosted}.

A notable advancement is PixelDP, which provides certified robustness guarantees for individual predictions, making it applicable in real-world scenarios where adversarial resilience is crucial \cite{lecuyer2019certifiedrobustnessadversarialexamples}. By ensuring reliable predictions amidst adversarial perturbations, PixelDP contributes to the development of secure AI systems.

The Latent Poisoning method enhances robustness by transforming the latent space of a variational autoencoder, effectively countering adversarial attacks through internal representation manipulation \cite{creswell2017latentpoisonadversarialattacks}. Such techniques underscore the importance of leveraging latent spaces to fortify model defenses against adversarial threats.

Adaptive adversarial training enhances traditional methods by introducing a two-step procedure that achieves asymptotic unbiasedness and variable-selection consistency, improving adversarial training performance and defense mechanisms \cite{xie2024asymptoticbehavioradversarialtraining}. This refinement ensures models withstand adversarial perturbations more effectively.

Additionally, addressing hidden incentives in machine learning systems is crucial for preventing adverse outcomes like filter bubbles and misinformation spread \cite{krueger2020hiddenincentivesautoinduceddistributional}. By diagnosing and managing these incentives, researchers can create models that are robust against adversarial attacks and aligned with ethical considerations, promoting trustworthy AI systems.

Collectively, these techniques enhance the robustness of machine learning models against adversarial examples, ensuring their reliability and security across diverse applications. By integrating these approaches, researchers can significantly improve AI systems' resilience to adversarial threats, leading to more dependable and secure AI applications.











\section{Uncertainty Estimation in Machine Learning} \label{sec:Uncertainty Estimation in Machine Learning}

Uncertainty estimation is fundamental in addressing the complexities inherent in machine learning. This section delves into various techniques for quantifying uncertainty, which are vital for improving the reliability and robustness of models, especially in dynamic data environments. By exploring these methods, we recognize their significance in ensuring models maintain performance amidst evolving data distributions. The following subsection details specific techniques for quantifying uncertainty, highlighting their distinct methodologies and implications.


\subsection{Methods for Quantifying Uncertainty} \label{subsec:Methods for Quantifying Uncertainty}

Quantifying uncertainty in machine learning models is crucial for enhancing their reliability in dynamic contexts. Recent studies indicate that incorporating atypicality into uncertainty quantification can boost the performance of neural networks and large language models, underscoring the necessity for structured education on these principles \cite{yuksekgonul2023confidencereliablemodelsconsider,valdenegrotoro2021teachinguncertaintyquantificationmachine}. Various methods have been devised, each providing unique strategies to measure and manage uncertainty. 

As illustrated in \autoref{fig:tiny_tree_figure_3}, the hierarchical classification of these methods is categorized into three main areas: adversarial robustness, optimization techniques, and interpretability methods. Each category encompasses specific methodologies, emphasizing their respective contributions to enhancing model reliability and interpretability.

PixelDefend is a notable method that enhances classifier resilience against adversarial attacks by using generative models to purify adversarial inputs, thereby improving uncertainty estimates \cite{song2018pixeldefendleveraginggenerativemodels}. Neural Collapse focuses on measuring metrics under different perturbations, demonstrating how understanding neural network geometry can enhance resilience to adversarial perturbations \cite{su2024robustnessneuralcollapseneural}.

In optimization, robust second-order methods reduce outlier influence, improving uncertainty estimates in nonconvex problems \cite{li2024robustsecondordernonconvexoptimization}. The ABCD-Conformal framework advances uncertainty modeling using Dropout and conformal prediction, ensuring robust predictions with guaranteed marginal coverage \cite{baragatti2024approximatebayesiancomputationdeep}. EasyUQ, based on isotonicity, preserves the order of uncertainty, ensuring consistent predictions across varying data distributions \cite{walz2023easyuncertaintyquantificationeasyuq}.

Interpretable methods, such as constructing prediction intervals for Random Forests using Out-of-Bag samples, provide valid coverage probabilities, enhancing model interpretability and reliability \cite{ramosaj2021interpretablemachinesconstructingvalid}. The TN-ML method employs von Neumann entropy to measure prediction certainty, particularly important in healthcare applications where accurate uncertainty estimation is crucial \cite{an2023intelligentdiagnosticschemelung}.

These methods collectively advance uncertainty estimation by utilizing statistical techniques like Categorical, Gaussian, and Dirichlet distributions, offering robust frameworks for accurately quantifying uncertainty in machine learning models \cite{valdenegrotoro2021teachinguncertaintyquantificationmachine}. By integrating these techniques, researchers can significantly enhance AI systems' reliability and informed predictions, ultimately increasing the trustworthiness and efficacy of AI applications.

\input{figs/tiny_tree_figure_3}
\subsection{Innovative Approaches to Uncertainty Estimation} \label{subsec:Innovative Approaches to Uncertainty Estimation}

Recent advancements have introduced innovative methods for uncertainty estimation, particularly in Out-of-Distribution (OOD) detection. These approaches employ probabilistic frameworks, such as Bayesian inference, to identify input samples outside the training distribution, which is crucial for the reliability of NLP systems in applications like text classification and machine translation \cite{lang2023survey,Out-of-Dis0,valdenegrotoro2021teachinguncertaintyquantificationmachine}.

Generative Adversarial Networks for Anomaly Detection (GAN-AD) is one innovative approach that models complex, non-linear data relationships. By using reconstruction errors alongside classification capabilities, GAN-AD provides a robust framework for uncertainty estimation and OOD detection \cite{li2019anomalydetectiongenerativeadversarial}.

The development of interpretable techniques, such as constructing valid prediction intervals for Random Forests using Out-of-Bag samples, ensures reliable and interpretable uncertainty estimates \cite{ramosaj2021interpretablemachinesconstructingvalid}. A structured approach to teaching uncertainty quantification categorizes existing research into practical use cases, facilitating learning and encouraging innovative solutions for specific OOD challenges \cite{valdenegrotoro2021teachinguncertaintyquantificationmachine}.

These innovative approaches, including statistical methods for modeling categorical, Gaussian, and Dirichlet distributions, significantly enhance uncertainty estimation, enabling models to effectively detect OOD samples and manage epistemic uncertainty. This improvement ultimately boosts their generalization beyond training data distributions \cite{valdenegrotoro2021teachinguncertaintyquantificationmachine}. By integrating these methods, researchers can enhance the reliability and trustworthiness of AI systems, leading to more effective applications across diverse domains.










\section{Challenges and Future Directions} \label{sec:Challenges and Future Directions}

The growing complexity and diversity of data across various applications have underscored significant challenges in Out-of-Distribution (OOD) detection methods. Understanding these challenges is crucial for identifying the limitations of current methodologies and informing future research. This section explores critical issues in existing OOD detection techniques that need to be addressed to improve their effectiveness across different domains.


\subsection{Challenges in Current OOD Detection Methods} \label{subsec:Challenges in Current OOD Detection Methods}

Current OOD detection methods face several challenges that impede their effectiveness. A primary issue is computational inefficiency, particularly in high-dimensional settings where techniques become impractical \cite{zhao2021suodacceleratinglargescaleunsupervised}. This is exacerbated by the reliance on multiple independent classifiers, which increases resource consumption and potential inaccuracies \cite{Out-of-Dis1}. Scalability is also a concern, as performance can degrade with large datasets or high anomaly rates \cite{li2021automateddataengineeringpipeline}. 

As illustrated in \autoref{fig:tiny_tree_figure_4}, the primary challenges faced by current OOD detection methods can be categorized into three main areas: computational inefficiency, data complexity, and bias and reliability issues. Each category highlights key aspects such as high-dimensional settings, non-linear correlations, and dataset bias that impede the effectiveness of these methods.

The complexity of non-linear correlations in multivariate time series data complicates OOD detection, especially due to the scarcity of labeled anomalies \cite{li2019anomalydetectiongenerativeadversarial}. Existing jamming techniques often fall short against advanced classifiers, creating further challenges \cite{kim2020make5gcommunicationsinvisible}. For instance, the TN-ML method highlights potential misclassification issues, necessitating human expert intervention \cite{an2023intelligentdiagnosticschemelung}.

Outliers significantly affect network embeddings, with many methods failing to account for them, leading to inaccurate network representations \cite{bandyopadhyay2018outlierawarenetworkembedding}. Techniques such as the Wεp distance may underperform in scenarios with extreme outlier contamination \cite{nietert2023outlierrobustoptimaltransportduality}. Furthermore, the reliability of prediction intervals is often questionable, as many methods lack theoretical guarantees for correct coverage probability \cite{ramosaj2021interpretablemachinesconstructingvalid}.

Bias in datasets is another critical hurdle, with existing methods inadequately addressing this issue, impairing models' ability to learn the true underlying tasks \cite{li2019repairremovingrepresentationbias}. The necessity for high-quality initial datasets and careful parameter tuning, as seen with UMAP, further limits applicability \cite{islam2024outlierdetectionlargeradiological}.

These challenges underscore the urgent need for more efficient, scalable, and robust OOD detection frameworks, particularly in NLP tasks. Addressing unique considerations such as discrete input spaces and contextual information is essential. A structured protocol for selecting both in-distribution (ID) and OOD datasets is necessary, emphasizing large and diverse ID datasets, distinct OOD distributions, and challenging distribution shifts that impact performance \cite{wang2024bridgingooddetectiongeneralization,lang2023survey,yuan2023revisiting}. By overcoming these limitations, researchers can enhance the reliability and applicability of OOD detection methods, contributing to more secure AI systems.

\input{figs/tiny_tree_figure_4}
\subsection{Future Directions in OOD Detection and Anomaly Detection} \label{subsec:Future Directions in OOD Detection and Anomaly Detection}

Future research in OOD detection and anomaly detection aims to address key challenges, enhancing machine learning model robustness and adaptability. A promising avenue is the exploration of domain generalization and multimodal data integration, which could significantly improve OOD detection effectiveness \cite{lang2023survey}. Leveraging diverse data sources will enable models to handle varying input conditions and generalize across domains.

In adversarial robustness, optimizing perturbation generation and applying these techniques in complex environments could lead to advancements in both OOD detection and adversarial strategies \cite{kim2020make5gcommunicationsinvisible}. This would enhance AI systems' resilience against sophisticated attacks in dynamic settings.

Refining statistical bounds and exploring robust method generalizations in high-dimensional inference tasks represent another critical research area \cite{nietert2023outlierrobustoptimaltransportduality}. Improving the theoretical foundations of these techniques will lead to more reliable OOD detection frameworks.

Adapting algorithms to dynamic networks, where outliers evolve, offers opportunities to enhance network embedding techniques \cite{bandyopadhyay2018outlierawarenetworkembedding}. Parallelizing these algorithms could further boost efficiency, enabling real-time processing of large-scale network data.

Future research should also focus on applying valid prediction interval methods to a broader range of machine learning techniques, addressing issues like missing values in covariates \cite{ramosaj2021interpretablemachinesconstructingvalid}. This could foster more transparent AI systems, enhancing trust in their predictions.

Additionally, providing theoretical results for clustering methods, such as Ward's Hierarchical clustering, and improving optimization for attack robustness remain important objectives \cite{chhabra2019suspicionfreeadversarialattacksclustering}. These efforts will bolster the robustness of clustering algorithms against adversarial manipulations.

Overall, these future directions highlight the potential for significant advancements in OOD and anomaly detection, contributing to the development of more secure, reliable, and adaptable AI systems. By addressing current limitations and exploring innovative methodologies, researchers can enhance AI models' generalization capabilities in dynamic environments.

\subsection{Challenges and Future Directions in Uncertainty Estimation} \label{subsec:Challenges and Future Directions in Uncertainty Estimation}

Uncertainty estimation in machine learning presents several challenges that limit the effectiveness of current methods. One major challenge is accurately quantifying uncertainty in high-dimensional settings, where data distribution complexities can obscure prediction reliability \cite{nietert2023outlierrobustoptimaltransportduality}. The presence of outliers complicates this task, as traditional methods often overlook their influence, resulting in inaccurate estimates \cite{nietert2023outlierrobustoptimaltransportduality}.

Integrating uncertainty estimation into dynamic and non-stationary environments is another significant challenge. Models must adapt to changing data distributions and provide reliable predictions under varying conditions, necessitating robust frameworks for effective uncertainty quantification \cite{walz2023easyuncertaintyquantificationeasyuq}.

Interpretability of uncertainty estimates is also crucial, as existing methods often lack transparency, making it difficult for users to trust AI predictions \cite{ramosaj2021interpretablemachinesconstructingvalid}. Developing interpretable uncertainty estimation techniques that provide clear insights into model predictions is essential.

Future research should focus on innovative approaches to address these challenges. Exploring multimodal data-driven methods that leverage diverse sources to enhance uncertainty estimates represents a promising direction \cite{lopez2023informativepriorsimprovereliability}. Integrating user behavior into uncertainty estimation frameworks, as seen in methods like SyRoUP, can provide nuanced insights into model predictions and improve reliability \cite{sicilia2024accountingsycophancylanguagemodel}.

Advancing robust optimization techniques that mitigate outlier influence and enhance uncertainty estimate stability is crucial for improving AI system reliability \cite{li2024robustsecondordernonconvexoptimization}. By tackling these challenges and exploring innovative methodologies, researchers can significantly enhance machine learning models' capability to provide reliable predictions, leading to more trustworthy AI applications.

\subsection{Emerging Trends and Technological Advancements} \label{subsec:Emerging Trends and Technological Advancements}

Emerging trends and technological advancements in OOD detection are set to enhance the robustness and adaptability of machine learning systems. A key trend is the integration of domain adaptation techniques, which improve model generalization when encountering OOD inputs \cite{lang2023survey}. This is particularly relevant in dynamic environments where models must adjust to varying data distributions.

Quantum machine learning advancements offer potential for handling complex data distributions more efficiently than classical methods. Techniques like Tensor-Network Machine Learning (TN-ML) demonstrate the application of quantum-inspired methods in healthcare, enhancing OOD detection interpretability and accuracy \cite{an2023intelligentdiagnosticschemelung}.

Developing robust optimization algorithms, such as Outlier-Robust Second-Order Optimization (ORSO), addresses challenges in handling corrupted data within stochastic optimization problems, enhancing model resilience against outliers and contributing to reliable OOD detection frameworks \cite{li2024robustsecondordernonconvexoptimization}.

In network analysis, adapting algorithms to dynamic networks with evolving outliers enhances network embedding applicability \cite{bandyopadhyay2018outlierawarenetworkembedding}. Parallelizing these algorithms could improve efficiency for real-time processing of large-scale network data.

Exploring multimodal data-driven methods that leverage diverse sources to enhance OOD detection robustness is another promising direction \cite{lopez2023informativepriorsimprovereliability}. These methods provide a comprehensive understanding of data distributions and improve OOD detection accuracy.

Additionally, integrating user behavior into OOD detection frameworks, as demonstrated by methods like SyRoUP, offers nuanced insights into model predictions, enhancing reliability \cite{sicilia2024accountingsycophancylanguagemodel}. This approach acknowledges the dynamic nature of user interactions and variability in model outputs.

Overall, these emerging trends and technological advancements indicate significant potential for improving OOD detection, contributing to the development of secure, reliable, and adaptable AI systems. By leveraging these innovations, researchers can enhance AI models' capability to generalize beyond training distributions, ensuring effectiveness in dynamic environments.











\section{Conclusion} \label{sec:Conclusion}

In this survey, we have explored the indispensable role of Out-of-Distribution (OOD) detection in fortifying AI systems against unforeseen inputs, enhancing their reliability and adaptability across various domains. The integration of OOD detection with related areas such as anomaly detection, adversarial robustness, and uncertainty estimation is pivotal in building AI systems capable of handling data beyond their training environments. This comprehensive approach ensures that AI models can maintain high performance and trustworthiness even when encountering atypical data.

In the realm of NLP, the ability to differentiate between human-generated and AI-generated content is increasingly critical. This capability not only supports transparency but also mitigates risks associated with misinformation and manipulation. The survey underscores the necessity of robust OOD detection frameworks to address these challenges effectively, emphasizing the importance of innovative methods that enhance model resilience and reliability.

As AI systems continue to evolve, the insights and methodologies discussed in this survey provide a foundation for advancing OOD detection techniques. By fostering a holistic understanding of these interrelated concepts, we can pave the way for more secure and dependable AI applications, ensuring their successful deployment in complex and dynamic environments.
